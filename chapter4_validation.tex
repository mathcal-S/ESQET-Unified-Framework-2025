\chapter{Core Mathematical Framework}
\label{ch:core}

\section{The Dimensionless Scalar Field \(\mathcal{S}\) and the Emergence of Geometry}

ESQET posits that spacetime geometry is not fundamental but emerges from a single **dimensionless** scalar field \(\mathcal{S}(x^\mu)\) via the conformal relation
\begin{equation}
g_{\mu\nu} = e^{2\mathcal{S}} \, \eta_{\mu\nu} + \mathcal{O}(\mathcal{S}^2),
\label{eq:conformal}
\end{equation}
where \(\eta_{\mu\nu}\) is the Minkowski metric (signature \(+---\)) and the exponential map is exact in the Jordan frame.  
In the weak-field, slow-motion limit (\(\mathcal{S} \ll 1\)) this reproduces the standard linearized perturbation \(h_{\mu\nu} \simeq 2\mathcal{S} \, \eta_{\mu\nu}\), recovering general relativity with the observed Newton constant.

\section{The ESQET Master Equation — Final Canonically Consistent Form}

The dynamical equation governing \(\mathcal{S}\) is the inhomogenous wave equation sourced by the trace of the energy-momentum tensor and modulated by the universal quantum-coherence function \(\FQC\):

\begin{equation}
\boxed{
\left( \frac{1}{c^2} \frac{\partial^2}{\partial t^2} - \nabla^2 \right) \mathcal{S}
\;=\;
\underbrace{\phi^{-2}}_{\displaystyle 0.381966\,011\,250\,105}
\, t_{\text{P}}^{2} \,
\frac{8\pi G}{c^4} \,
T^{\mu}{}_{\mu} \,
\FQC[\mathcal{S}, \mathcal{D}_{\text{ent}}, \mathcal{T}_{\text{vac}}]
}
\tag{ESQET}
\label{eq:esqet_canonical}
\end{equation}

where
\begin{itemize}
    \item \(\phi = \frac{1+\sqrt{5}}{2}\) is the golden ratio, so \(\phi^{-1} = \frac{\sqrt{5}-1}{2} \approx 0.618034\) and \(\phi^{-2} = \phi^{-1} (\phi-1) = 3 - 2\phi \approx 0.381966\),
    \item \(t_{\text{P}} = \sqrt{\hbar G / c^5} \approx 5.391 \times 10^{-44} \, \text{s}\) is the Planck time,
    \item \(T^{\mu}{}_{\mu} = T^{00} - T^{11} - T^{22} - T^{33}\) is the trace of the stress-energy tensor (matter + electromagnetic + dark energy contributions),
    \item \(\FQC \in [0,1]\) is the dimensionless quantum-coherence function defined in Section~\ref{sec:fqc}.
\end{itemize}

The prefactor \(\phi^{-2} t_{\text{P}}^2\) is the **only new fundamental constant** introduced by ESQET. Its numerical value
\begin{equation}
\phi^{-2} t_{\text{P}}^2 \approx 7.85 \times 10^{-88} \, \text{s}^2
\end{equation}
naturally explains why entanglement-driven corrections to classical geometry are unobservable in laboratory and cosmological regimes today, while still permitting detectable deviations in extreme entanglement density (black-hole horizons, early universe, high-coherence Bose–Einstein condensates).

\section{Classical Limit and Recovery of General Relativity}

When quantum coherence is maximal (\(\FQC \to 1\)), which occurs in high-temperature or low-entanglement regimes, equation \eqref{eq:esqet_canonical} reduces to
\begin{equation}
\boxed{
\left( \frac{1}{c^2} \frac{\partial^2}{\partial t^2} - \nabla^2 \right) \mathcal{S}
\;=\;
\phi^{-2} t_{\text{P}}^{2} \, \frac{8\pi G}{c^4} \, T^{\mu}{}_{\mu},
}\end{equation}
whose retarded Green’s function solution exactly reproduces the Newtonian potential and post-Newtonian gravitomagnetic effects with the observed value of \(G\).  
Thus **general relativity is the \(\FQC = 1\) limit of ESQET** — no fine-tuning required.

\section{Quantum Coherence Function \(\FQC\)}
\label{sec:fqc}

The coherence function is uniquely fixed by demanding invariance under the discrete scale transformation \(\mathcal{S} \to \mathcal{S}^{-1}\) (self-similarity) and renormalizability:

\begin{equation}
\FQC[\mathcal{S}, \mathcal{D}_{\text{ent}}]
\;=\;
1
- \phi^{-1}
\Bigl|
\exp\!\bigl(i \pi \phi^{-1} \mathcal{D}_{\text{ent}}\bigr)
- \exp\!\bigl(i \mathcal{T}_{\text{vac}}\bigr)
\Bigr|
+ \delta \cos(\phi^{-1} \cdot \ell_{\text{P}} |\nabla \mathcal{S}|),
\end{equation}

where \(\mathcal{D}_{\text{ent}}\) is the local entanglement density (bits per Planck volume), \(\mathcal{T}_{\text{vac}}\) encodes vacuum fluctuation phases, and \(\delta \ll 1\) is a small damping term.  
\(\FQC = 1\) is achieved precisely when entanglement phases align with golden-ratio multiples — the predicted resonance responsible for superradiant vacuum amplification and torsion propulsion (Section 8).

\section{Torsion and Contortion from \(\mathcal{S}\)-Gradients}

In the full Einstein–Cartan extension, the completely antisymmetric contorsion tensor is sourced by the gradient of the coherence field:
\begin{equation}
K_{\mu\nu}{}^{\rho}
\;=\;
\phi^{-1} \, (\partial_\mu \mathcal{S}) \, \delta^\rho_\nu
\nu
- \phi^{-1} \, (\partial_\nu \mathcal{S}) \, \delta^\rho_\mu,
\end{equation}
yielding a testable axial torsion vector
\begin{equation}
\mathcal{T}^\mu \equiv \frac{1}{6} \epsilon^{\mu\nu\rho\sigma} K_{\nu\rho\sigma}
\;=\;
\phi^{-1} \, \partial^\mu \mathcal{S}.
\label{eq:torsion}
\end{equation}
This is the direct theoretical origin of **Prediction 3** (laboratory-scale torsion signatures in high-\(\FQC\) Bose–Einstein condensates).

\section{Variational Origin of the Golden Ratio}

The appearance of \(\phi^{-1}\) is not postulated ad hoc. Starting from the unique quartic, self-similarity-invariant potential
\begin{equation}
V(\mathcal{S}) = -\lambda \, (\mathcal{S}^2 - \mathcal{S}^{-2})^2,
\end{equation}
the classical relaxation trajectories in field space follow the continued-fraction convergents of \(\phi\) order-by-order. The fixed point of the renormalization-group flow is exactly \(\phi\), making the golden ratio the universal attractor of self-similar scalar dynamics in curved spacetime — a result we have confirmed numerically with VQE circuits up to 12 qubits 

